\section{Karten}
\subsection{FailCard}
\paragraph{Projektbeschreibung} Kleine (Vistenkarten ähnliche) Karten, die man Personen, die gerade etwas sehr doofes getan haben, in die Hand drücken kann, um zu zeigen, wie sehr man an dem Unglück anteilnimmt.
\paragraph{Textbausteine}
Die Folgenden Textbausteine sollen alle auf die Karten. Weitere Übersetzungen gerne gesehen. (Bitte keinen Google-Translate)
\begin{itemize}
	\item Fühle dich mit dem Erhalt dieser Karte über dein episches Versagen informiert.
	\item If you receive this card, please feel informed that you have failed in an epic way.
	\item Si has recibido esta tarjeta significa que la has liado parda.
	\item Si vous avez reçu cette carte, ca veut dire que vous fautez épiquement.
	\item Kung makatanggap ka ng card na ito, sana ay malaman mo na sobrang pumalpak ka.
	\item Osečaj se sa prijemom ove karte obaveštenim, o tvom epsokom newuspehu.
	\item putem oue karte budi informiran o tovm epskom neuspjehu.
	\item po obdrzezi teto karty, citim s tebou a jsem irformavozy pries touje ruzre myslensky.
	\item \foreignlanguage{russian}{oceђaj le ca npaиjemon obe kapte, o?abeшtheиm, o tbom eпckom heуспxy.}
	\item \foreignlanguage{russian}{получаването на тази карта, се чувствай информиран за твоя епически провал.}
	\item \foreignlanguage{russian}{poczuj si? po otrzymaniu tej karty ?ako ten który zawali? na ca?ej elini.}
\end{itemize}

\subsection{Mosfetkiller Quartett}
\paragraph{Projektbeschreibung} Ein Quartett mit den aktivieren Usern das Forums in möglichst vorurteilsbehafteter Darstellung. 
\paragraph{Bewertungskriterien} Nach folgenden Kriterien sollten die Karten aufgebaut sein.
\begin{itemize}
	\item Sinnvolle Forenposts
	\item ...
\end{itemize}

\paragraph{Produktionsdienstleister} Eine Möglichkeit könnte die Nutzung von \url{http://anybuddy.org}
