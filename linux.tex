% don't forget to build it twice %
\documentclass[oneside,12pt,a4paper]{scrartcl}
\usepackage[OT2, T1]{fontenc}
\usepackage[utf8]{inputenc}
\usepackage[russian, ngerman]{babel}

\usepackage[bottom=4cm]{geometry}
\usepackage[bottom]{footmisc}
\usepackage{fancyhdr}
\usepackage{lastpage}

\usepackage{amsmath}
\usepackage{amsfonts}
\usepackage{amssymb}

\usepackage{graphicx}
\usepackage{color}
\usepackage{hyperref}
\usepackage{listings}
\usepackage{enumitem}
\usepackage{cmap}
\usepackage{tabularx}

\fancypagestyle{plain}{
	\renewcommand{\footrulewidth}{1pt}
	\renewcommand{\headrulewidth}{1pt}
	\fancyhf{}
	\fancyhead[R]{\#mosfetkiller @ irc.rizon.net}
	\fancyhead[L]{\today}
	\fancyhead[C]{Wissenssammlung}
	\fancyfoot[C]{\thepage/\pageref{LastPage}}
}
\pagestyle{plain}

\hypersetup{
    colorlinks=true, 
    linktoc=all,    
    allcolors=black, 
    urlcolor=blue
}

\title{Sammlung nützlicher Inhalte}
\author{\href{http://chat.mibbit.com/?channel=\%23mosfetkiller\&nick=your_nick_here\&server=irc.rizon.net\&autoConnect=true}{\#mosfetkiller@irc.rizon.net:6667}}
\newcommand{\linkitem}[1]{\item \url{#1}}


\begin{document}
\maketitle

\tableofcontents
\clearpage

\part{Technisches}
\section{Distributionsauswahl}
\begin{itemize}
\linkitem{http://distrowatch.com/}
\linkitem{http://www.zegeniestudios.net/ldc/index.php?lang=de}
\end{itemize}


\section{(Linux)Administration}
\begin{itemize}
\linkitem{http://daemonkeeper.net/70/dein-neuer-linux-server/}
\linkitem{http://www.forwiss.uni-passau.de/~berberic/klickop.html}
\end{itemize}
	\subsection{UDev}
	\begin{itemize}
	\linkitem{http://www.reactivated.net/writing_udev_rules.html}
	\linkitem{http://wiki.debian.org/Persistent_disk_names}
	\end{itemize}
	
	\subsection{systemd}
	\begin{itemize}
	\linkitem{http://ewontfix.com/14/}
	\linkitem{http://www.phoronix.com/scan.php?page=news_item&px=MTczNDk}
	\linkitem{http://www.phoronix.com/scan.php?page=news_item&px=MTczNjI}
	\end{itemize}


\section{Linux im Allgemeinen}
\begin{itemize}
\linkitem{http://daemonkeeper.net/43/linux-ist-nichts-fuer-dich-lass-es/}
\linkitem{http://linux.oneandoneis2.org/LNW.htm}
\end{itemize}

\section{IRC}
\begin{itemize}
\linkitem{http://sackheads.org/~bnaylor/spew/away_msgs.html}
\linkitem{http://irssi.org/}
\end{itemize}

\section{Memes}
\begin{itemize}
%we should cache them to prevent disapperance %
%we should. now.% 
\linkitem{http://cdn.memegenerator.net/instances/400x/15098309.jpg}
\linkitem{http://filestore.dasnet.bplaced.de/img/mem.png}
\linkitem{http://genetic-defect.eu/hammer.png}
\linkitem{http://cdn.memegenerator.net/instances/400x/37429472.jpg}
\linkitem{http://memecrunch.com/meme/KFBU/linus-torvalds-verachtet-dich/image.png}
\linkitem{https://raw.githubusercontent.com/jwacalex/linux_wiki/master/Graph/other/SystemD.jpg}
\end{itemize}

\section{Programmierung}
\begin{itemize}
\linkitem{http://www.infosun.fim.uni-passau.de/cl/lehre/ppp2013/hinweise_zu_c.html}
\linkitem{http://yorickpeterse.com/articles/use-bcrypt-fool/}
\linkitem{https://wiki.theory.org/YourLanguageSucks}
\end{itemize}

\subsection{PHP}
\begin{itemize}
\linkitem{http://phpsadness.com/}
\linkitem{http://me.veekun.com/blog/2012/04/09/php-a-fractal-of-bad-design/}
\linkitem{http://habnab.it/php-table.html}
\linkitem{http://www.ozerov.de/bigdump/}
\end{itemize}

\subsection{C}
\begin{itemize}
\linkitem{https://www.youtube.com/watch?v=eJ7HP7fpnW8}
\end{itemize}


\section{\glqq Nützliche \grqq Links}
\begin{itemize}
\linkitem{http://badum-tish.com/}
\linkitem{http://www.dramabutton.com/}
\end{itemize}


\section{\LaTeX}
\begin{itemize}
\linkitem{https://github.com/jwacalex/linux\_wiki/blob/master/Makefile}
\linkitem{http://www.literatur-generator.de/}
\end{itemize}



\part{Lebenswichtiges}
\section{Rezepte}
\subsection{Fondue}
Fondue à la Doeme:

\subsubsection{Zutaten}
Für eine Person:
\begin{itemize}
\item 125g Vacherin
\item 125g Greyerzer
\item Weisswein
\item Kirschwasser
\item Maisstärke
\item Knoblauch
\end{itemize}
\subsubsection{Rezept}

\begin{enumerate}
\item Fonduecaquelon~\footnote{Eine Pfanne} mit ein paar Knoblauchzehen ausreiben.
\item Caquelon mit dem Knoblauch etwas erhitzen, und Weisswein zugeben (ca. 3mm $\frac {\mbox{Füllhöhe}} {\mbox{Person}}$ bei normaler Caquelondimension [$\sim$22cm])~\footnote{Etwas Weisswein}.
\item Nach weiterem kurzen Erwärmen den geriebenen Käse zugeben.
\item Die Mischung mit Rührbewegungen der Form einer 8 folgend zum globalen Schmelzpunkt begleiten~\footnote{Rühren bis die Brühe flüssig ist.}.
\item Sobald das Fondue flüssig ist, Maisstärke in einem Glässchen Kirschgeist zu einer Suspension verrühren, und selbige anschliessend dem Fondue beifügen.
\item Den Fonduecaquelon geographisch auf den Rechaud replazieren.
\item \emph{Solange das Fondue erwärmt wird, muss es permanent mit einem Löffel oder Löffelersatz~\footnote{Gabel mit Brotaufsatz} gerührt werden.} 
\end{enumerate}

\subsection{Met-Hollunder-Schorle}
\begin{itemize}
\item $2^{0}$ Anteile Hollundersirup
\item $2^{2}$ Anteile Met
\item $2^{3}$ Anteile Mineralwasser, spritzig
\end{itemize}
\subsection{Gute heisse Schokolade}
\label{sec:hot_chocolate}
Mengen müssen selber evaluiert werden!
\begin{itemize}
	\item Milch.~\footnote{\url{http://www.youtube.com/watch?v=w4aLThuU008}}
	\item Kakaopulver(Kein Schokoladenpulver, das reine).
	\item Mascobadozucker, bzw. brauner Zucker.
	\item Schokoladenpulver.
\end{itemize}
Das Verhältnis von $\frac{\mbox{Schokoladenpulver}}{\mbox{Kakaopulver}+\mbox{Mascobadozucker}}$ sollte unter $\frac{1}{1}$ liegen, muss aber wie bereits erwähnt selbst ermittelt werden.
\subsection{Chocolat Rum}
\begin{itemize}
	\item Schokolade gemäss Kapitel~\ref{sec:hot_chocolate}, Seite~\pageref{sec:hot_chocolate}.
	\item Einen Schuss Rum.
	\item Ein wortwörtliches Sahnehäubchen.
\end{itemize}
\section{*.Körper}
\subsection{Haare}
\subsubsection{Kokosöl}
Wenn die Haare sehr trocken sind/neigen sich zu verknoten, kann es helfen, die Längen vor dem Duschen mit Kokosöl (echtes) einzufetten. Alternativ ins trockene Haar eine sehr kleine Menge einarbeiten.

\part{Kommunkatives}
\section{Richtig Fragen stellen}
\begin{itemize}
\linkitem{http://www.catb.org/esr/faqs/smart-questions.html}
\end{itemize}

\section{Richtig Diskutieren}
\begin{itemize}
\linkitem{https://yourlogicalfallacyis.com}
\linkitem{http://de.wikipedia.org/wiki/Eristische_Dialektik}
\linkitem{http://www.ratioblog.de/archive/Fehlschl%FCsse}
\linkitem}{http://rationalwiki.org/wiki/Category:Fallacious_arguments}
\end{itemize}


\part{Projekte}
\section{Karten}
\subsection{FailCard}
\paragraph{Projektbeschreibung} Kleine (Vistenkarten ähnliche) Karten, die man Personen, die gerade etwas sehr doofes getan haben, in die Hand drücken kann, um zu zeigen, wie sehr man an dem Unglück anteilnimmt.
\paragraph{Textbausteine}
Die Folgenden Textbausteine sollen alle auf die Karten. Weitere Übersetzungen gerne gesehen. (Bitte keinen Google-Translate)
\begin{itemize}
	\item Fühle dich mit dem Erhalt dieser Karte über dein episches Versagen informiert.
	\item If you receive this card, please feel informed that you have failed in an epic way.
	\item Si has recibido esta tarjeta significa que la has liado parda.
	\item Si vous avez reçu cette carte, ca veut dire que vous fautez épiquement.
	\item Kung makatanggap ka ng card na ito, sana ay malaman mo na sobrang pumalpak ka.
	\item Osečaj se sa prijemom ove karte obaveštenim, o tvom epsokom newuspehu.
	\item putem oue karte budi informiran o tovm epskom neuspjehu.
	\item po obdrzezi teto karty, citim s tebou a jsem irformavozy pries touje ruzre myslensky.
	\item \foreignlanguage{russian}{oceђaj le ca npaиjemon obe kapte, o?abeшtheиm, o tbom eпckom heуспxy.}
	\item \foreignlanguage{russian}{получаването на тази карта, се чувствай информиран за твоя епически провал.}
	\item \foreignlanguage{russian}{poczuj si? po otrzymaniu tej karty ?ako ten który zawali? na ca?ej elini.}
\end{itemize}

\subsection{Mosfetkiller Quartett}
\paragraph{Projektbeschreibung} Ein Quartett mit den aktivieren Usern das Forums in möglichst vorurteilsbehafteter Darstellung. 
\paragraph{Bewertungskriterien} Nach folgenden Kriterien sollten die Karten aufgebaut sein.
\begin{itemize}
	\item Sinnvolle Forenposts
	\item ...
\end{itemize}

\paragraph{Produktionsdienstleister} Eine Möglichkeit könnte die Nutzung von \url{http://anybuddy.org}


\end{document}
