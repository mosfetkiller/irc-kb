% don't forget to build it twice %
\documentclass[oneside,12pt,a4paper]{scrartcl}
\usepackage[OT2, T1]{fontenc}
\usepackage[utf8]{inputenc}
\usepackage[russian, ngerman]{babel}

\usepackage[bottom=4cm]{geometry}
\usepackage[bottom]{footmisc}
\usepackage{fancyhdr}
\usepackage{lastpage}

\usepackage{amsmath}
\usepackage{amsfonts}
\usepackage{amssymb}

\usepackage{graphicx}
\usepackage{color}
\usepackage{hyperref}
\usepackage{listings}
\usepackage{enumitem}
\usepackage{cmap}
\usepackage{tabularx}

\newcommand{\irc}{\href{http://chat.mibbit.com/?channel=\%23mosfetkiller\&nick=your_nick_here\&server=irc.rizon.net\&autoConnect=true}{\#mosfetkiller@irc.rizon.net:6667}}
\author{\irc}
\newcommand{\about}{Sammlung nützlicher Inhalte}
\title{\about}

\fancypagestyle{plain}{
	\renewcommand{\footrulewidth}{1pt}
	\renewcommand{\headrulewidth}{1pt}
	\fancyhf{}
	\fancyhead[R]{\irc}
	\fancyhead[L]{\about}
	\fancyhead[C]{}
	\fancyfoot[R]{Seite \thepage/\pageref{LastPage}}
	\fancyfoot[L]{\today}
}
\pagestyle{plain}

\hypersetup{
    colorlinks=true, 
    linktoc=all,    
    allcolors=black, 
    urlcolor=blue
}



\newcommand{\linkitem}{\item \url}

\newcommand{\enq}[1]{,,#1''}


\begin{document}
\maketitle
\tableofcontents
\clearpage

\newpage
\part{Technisches}
\input{./parts/technical.tex}

\newpage
\part{Lebenswichtiges}
\section{Rezepte}
\subsection{Fondue}
Fondue à la Doeme:

\subsubsection{Zutaten}
Für eine Person:
\begin{itemize}
\item 125g Vacherin
\item 125g Greyerzer
\item Weisswein
\item Kirschwasser
\item Maisstärke
\item Knoblauch
\end{itemize}
\subsubsection{Rezept}

\begin{enumerate}
\item Fonduecaquelon~\footnote{Eine Pfanne} mit ein paar Knoblauchzehen ausreiben.
\item Caquelon mit dem Knoblauch etwas erhitzen, und Weisswein zugeben (ca. 3mm $\frac {\mbox{Füllhöhe}} {\mbox{Person}}$ bei normaler Caquelondimension [$\sim$22cm])~\footnote{Etwas Weisswein}.
\item Nach weiterem kurzen Erwärmen den geriebenen Käse zugeben.
\item Die Mischung mit Rührbewegungen der Form einer 8 folgend zum globalen Schmelzpunkt begleiten~\footnote{Rühren bis die Brühe flüssig ist.}.
\item Sobald das Fondue flüssig ist, Maisstärke in einem Glässchen Kirschgeist zu einer Suspension verrühren, und selbige anschliessend dem Fondue beifügen.
\item Den Fonduecaquelon geographisch auf den Rechaud replazieren.
\item \emph{Solange das Fondue erwärmt wird, muss es permanent mit einem Löffel oder Löffelersatz~\footnote{Gabel mit Brotaufsatz} gerührt werden.} 
\end{enumerate}

\subsection{Met-Hollunder-Schorle}
\begin{itemize}
\item $2^{0}$ Anteile Hollundersirup
\item $2^{2}$ Anteile Met
\item $2^{3}$ Anteile Mineralwasser, spritzig
\end{itemize}
\subsection{Gute heisse Schokolade}
\label{sec:hot_chocolate}
Mengen müssen selber evaluiert werden!
\begin{itemize}
	\item Milch.~\footnote{\url{http://www.youtube.com/watch?v=w4aLThuU008}}
	\item Kakaopulver(Kein Schokoladenpulver, das reine).
	\item Mascobadozucker, bzw. brauner Zucker.
	\item Schokoladenpulver.
\end{itemize}
Das Verhältnis von $\frac{\mbox{Schokoladenpulver}}{\mbox{Kakaopulver}+\mbox{Mascobadozucker}}$ sollte unter $\frac{1}{1}$ liegen, muss aber wie bereits erwähnt selbst ermittelt werden.
\subsection{Chocolat Rum}
\begin{itemize}
	\item Schokolade gemäss Kapitel~\ref{sec:hot_chocolate}, Seite~\pageref{sec:hot_chocolate}.
	\item Einen Schuss Rum.
	\item Ein wortwörtliches Sahnehäubchen.
\end{itemize}
\subsection{Tassenbrownie}
Ein Brownie mit hohem Studentenfaktor:
\begin{itemize}
	\item 3 EL Mehl
	\item 1 EL Gemahlene Haselnüsse
	\item 3 EL Kristallzucker
	\item 2 EL Kakaopulver
	\item Eine Prise Salz
	\item Obige Zutaten in einer Tasse möglichst homogen mischen.
	\item Eventuell ein paar Schokostücke
	\item Eventuell ein paar Walnussstücke
	\item 2 EL Sonnenblumenöl
	\item 2-3 EL Milch. Wasser geht, für die Laktoseintoleranten unter uns, auch.
	\item Gut umrühren.
	\item Tasse in die Mikrowelle und auf höchster Stufe laufen lassen.
	Dauer: ca. eine Minute, jedoch variiert das stark je nach Mikrowelle.
	Heuristik: So lange kochen, bis man Schokoladengeruch riecht, danach nochmal ca. 10 Sekunden.
\end{itemize}

Den Brownie heiss essen, denn er wird beim Auskühlen schnell trocken.
\section{*.Körper}
\subsection{Haare}
\subsubsection{Kokosöl}
Wenn die Haare sehr trocken sind/neigen sich zu verknoten, kann es helfen, die Längen vor dem Duschen mit Kokosöl (echtes) einzufetten. Alternativ ins trockene Haar eine sehr kleine Menge einarbeiten.

\newpage
\part{Kommunkatives}
\section{Richtig Fragen stellen}
\begin{itemize}
\linkitem{http://www.catb.org/esr/faqs/smart-questions.html}
\end{itemize}

\section{Richtig Diskutieren}
\begin{itemize}
\linkitem{https://yourlogicalfallacyis.com}
\linkitem{http://de.wikipedia.org/wiki/Eristische_Dialektik}
\linkitem{http://www.ratioblog.de/archive/Fehlschl%FCsse}
\linkitem{http://rationalwiki.org/wiki/Category:Fallacious_arguments}
\end{itemize}


\newpage
\part{Projekte}
\input{./parts/projects.tex}

\end{document}
